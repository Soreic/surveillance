\Abstract{
The availability of geocoded health data and the inherent temporal
structure of communicable diseases have led to the development of
so-called two-component endemic-epidemic statistical models
for the spatio-temporal analysis of infectious disease occurrence.
The present work consists of an overview of such models and describes
functionality of the open source \proglang{R} package \pkg{surveillance}
to analyse data typically collected as part of routine public health
surveillance systems.

We distinguish data by its available resolution in space and time.
Individual infections occur continuously in time and continuously in space,
whereas for epidemics across households or farms the set of possible event
locations is discrete.
Such small-scale event data is modelled by variations of Hawkes self-exciting
point-process consisting of additive endemic and epidemic components
parametrized by covariates.
% Maximum likelihood estimation for these models is implemented in functions
% \code{twinstim} and \code{twinSIR}.
If data are available in temporally and geographically aggregated form,
a multivariate time-series model for counts can be applied,
which is similarly decomposed into endemic and epidemic components.
% the \code{S4} class \code{"sts"} can be used to represent data and the function
% \code{hhh4} allows for similar endemic-epidemic time-series modeling as above.
% Here, spatial correlation in the time series can also be covered by additional
% random effects in a conditional autoregressive formulation.

Methods and implementations are illustrated by data on invasive
meningococcal disease in Germany 2002--2008, the Hagelloch 1861 measles
outbreak, and influenza in Southern Germany 2001--2008.
Altogether, the package \pkg{surveillance} contains a comprehensive set of tools for
spatio-temporal analyses of epidemic phenomena, which also comprise forest
fires, earthquakes or criminal events.
}

\Keywords{space-time modeling, infectious diseases, epidemics,
          point process models, multivariate time series of counts}
%\Plainkeywords{keywords, comma-separated, not capitalized, Java} %% without formatting
%% at least one keyword must be supplied
